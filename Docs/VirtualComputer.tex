\documentclass[12pt]{article}
\usepackage[utf8]{inputenc}
\usepackage{url}

\title{VirtualComputer}
\author{RootCellar}
\date{\today}
\begin{document}
\maketitle

\section{Introduction}

This program is an implementation of virtual hardware that can be programmed with a language similar to assembly.
It includes emulation of the CPU and RAM, and may be expanded in the future to have a disk, and even internet access.
The language itself is intended to be an easier form of assembly, that makes for a great programming challenge, while
removing unnecessary difficulties.

\section{Assembler}

\vspace{12pt}

Each instruction, after being assembled, is 13 bytes broken up in the following way:
\begin{itemize}

\item
1 Byte: Instruction Code/ID

\item
4 Bytes: Parameter 1

\item
4 Bytes: Parameter 2

\item
4 Bytes: Location of the next instruction to execute

\end{itemize}

\vspace{12pt}

The instruction code is simply a number that the CPU uses to figure out what to do with the given parameters.
The next instruction location bytes are used to tell the CPU which instruction (by location) should be executed next.
All instructions are the same size, regardless of whether or not all of the space is used.
In the case of IF statements, the assembler will use Parameter 2 as a secondary next instruction location
in case that the condition fails.

It is possible to write more powerful programs by setting those four numbers manually. For instance,
the assembler will not allow you to write data to instructions. But, when entering parameters manually and using PUT or MOV, it is possible
to write data to an instruction, allowing for code injection from inside the program.

By setting making the instructions manually, it is also possible to write faster programs.
This is mostly in the case of setting the next instruction location manually, to avoid wasting CPU cycles.
For example: if the assembler outputs code that makes the CPU execute an instruction and then call a GOTO,
you could instead speed this up manually by removing the goto and simply telling the previous instruction where
the CPU should go next. The assembler is intended to attempt these small optimizations itself, but it may not always succeed.

\section{The Language}

\end{document}
