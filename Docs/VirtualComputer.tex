\documentclass[12pt]{article}
\usepackage[utf8]{inputenc}
\usepackage{url}

\title{\bfseries VirtualComputer}
\author{RootCellar}
\date{\today}
\begin{document}
\maketitle

Github Repository: \url{https://Github.com/RootCellar/VirtualComputer}

\section{Introduction}

This program is an implementation of virtual hardware that can be programmed with a language similar to assembly.
It includes emulation of the CPU and RAM, and may be expanded in the future to have a disk, and even internet access.
The language itself is intended to be an easier form of assembly that makes for a great programming challenge, while
removing unnecessary difficulties.

\section{Using the Program: Linux}

Using the program on Linux is very easy, especially if you're working from within a clone of the git repository.
The folder called "Scripts" has bash scripts that make both development, and use of the program much easier.
Before you run the scripts, you may need to run "chmod u+x" or "chmod a+x" from inside the Scripts folder to
ensure you have permission to run the scripts.

When you are able to run the scripts, building the program is as easy as ./build.sh in Scripts. Note that this requires a Java compiler,
and will in-turn require Java to run the program. After building, you may want to try one of the example assembly
files to make sure that the program works. You can try "Test.vasm" in "Examples" in two ways: either copy Examples/Test.vasm to build, or
run "./execasm.sh ../Examples/Test" in Scripts. If you copy the source file, you can just run ./execasm.sh.

Once the assembly source file is assembled, it can be executed with ./execmain.sh in Scripts. Note that this will open a GUI,
which can be disabled by supplying the -nogui option. The virtual computer will then start up and run run.vbin in the folder "build".
Assuming all goes well, your program should then execute until it is complete!

The other scripts are used mostly for development and information tracking. "info.sh" will collect information and place it
in multiple files in the folder "info". This includes information such as line count, comments (and TODOs), and even a sorted list of file sizes.

\vspace{12pt}

Other useful scripts include:

\begin{itemize}

\item
"jar.sh" is used to package the entire program into one singular jar file, to make for something more portable.

\item
"rm.sh" will remove .temp files, .class files, and .jar files.

\item
"tail.sh" and "tailasm.sh" are used to follow the verbose logs in "build/Logs".
"tail.sh" follows the log while the virtual computer is running,
whereas "tailasm.sh" follows the assembler logs.
These are used to find bugs in the program.

\item
"builddoc.sh" is used to build this document!

\end{itemize}

A small warning: the "Logs" folder in "build" can become pretty sizable if you use the program a lot.
You may want to delete these every so often. Running rm.sh in Scripts will clear these.

A small note: supplying an integer argument to the program allows you to set the CPU speed in Hertz (Hz)

\section{Assembler}

\vspace{12pt}

Each instruction, after being assembled, is 13 bytes broken up in the following way:
\begin{itemize}

\item
1 Byte: Instruction Code/ID

\item
4 Bytes: Parameter 1

\item
4 Bytes: Parameter 2

\item
4 Bytes: Location of the next instruction to execute

\end{itemize}

\vspace{12pt}

The instruction code is simply a number that the CPU uses to figure out what to do with the given parameters.
The next instruction location bytes are used to tell the CPU which instruction (by location) should be executed next.
All instructions are the same size, regardless of whether or not all of the space is used.
In the case of IF statements, the assembler will use Parameter 2 as a secondary next instruction location
in case that the condition fails.

It is possible to write more powerful programs by setting those four numbers manually. For instance,
the assembler will not allow you to write data to instructions. But, when entering parameters manually and using PUT or MOV, it is possible
to write data to an instruction, allowing for code injection from inside the program.

By making the instructions manually, it is also possible to write faster programs.
This is mostly in the case of setting the next instruction location manually, to avoid wasting CPU cycles.
For example: if the assembler outputs code that makes the CPU execute an instruction and then call a GOTO,
you could instead speed this up manually by removing the GOTO and simply telling the previous instruction where
the CPU should go next. The assembler is intended to attempt these small optimizations itself, but it may not always succeed.

\section{The Language}

\begin{itemize}

\item
NOOP: Do nothing. This still consumes a clock cycle.

\item
PUT: Put hard-coded data somewhere

\item
MOV: Move one piece of data somewhere else

\item
LABEL: Define a label that the program can go to later

\item
GOTO: go to a label

\item
DATA: Define a variable

\item
IF: Check if a piece of data is true

\item
ELIF: Else if

\item
ELSE: Do this if the previous if was false

\item
END: The end of an if-then-else block

\item
EQ: Check of the data in the register equals the given value

\item
LEAQ

\item
GREQ

\item
GREATER

\item
LESS

\item
ADD

\item
SUBTRACT

\item
MULTIPLY

\item
DIVIDE

\item
POW

\item
MOD

\item
LSHIFT

\item
RSHIFT

\item
AND

\item
OR

\item
XOR

\item
EXIT: End execution of the program

\item
DISPVAL: Display a hard coded value

\item
DISPLOC: Display data at a location (register/RAM)

\end{itemize}

\section{Debugging your program}

While working on your programs, you may have times where your program doesn't work properly,
and you aren't sure of why. The "tailasm.sh" and "tail.sh" scripts in "Scripts" may help you
in this instance, as they allow you to follow the assembling and executing processes as they
happen. You can also just read the logs.

Using these scripts to help you debug your program will likely require that you understand
the VirtualComputer very well. They are extremely useful tools for both debugging the
VirtualComputer program itself, and the programs that run on it.

\section{Notes/Disclaimers}

The creator of this project is RootCellar (\url{https://Github.com/RootCellar})

\begin{itemize}

\item
This document only specifies the default language.

\item
The information in this document may not always be up to date.

\item
Relating to the previous point, this program is intended to be mod-able.
It isn't currently, however this document may not accurately describe other hardware and
language implementations.

\item
The default programming language does not currently have a name. Ideas welcomed.

\item
The default hardware components also do not have names.

\end{itemize}

\end{document}
